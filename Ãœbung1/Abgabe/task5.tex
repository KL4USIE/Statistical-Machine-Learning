\exercise{Bayesian Decision Theory}
\begin{questions}
	
	%----------------------------------------------
	\begin{question}{Optimal Boundary }{4}		
	\begin{answer} 
			\begin{enumerate}
		\item 
		Bayesian decision theory is a fundamental statistical approach to the problem of pattern classification based probabilities.
		\item The goal is to decide which class an example $x$ most likely belongs to.
		This is done by comparing the class posterior probabilities $p(C_i\mid x)$, which can be calculated by Bayes' theorem:
		\[
		p(C_i \mid x)=\frac{p(x\mid C_i)p(C_i)}{p(x)} \propto p(x\mid C_i)p(C_i)
		\]
		\item The decision boundary of two classes $C_1$ and $C_2$ is given by $p(C_1\mid x)=p(C_2\mid x)$, where  $C_1$ is chosen over $C_2$ if $p(C_1\mid x)>p(C_2\mid x)$.
		\end{enumerate}
		
	\end{answer}
		
	\end{question}
	
		%----------------------------------------------
	\begin{question}{Decision Boundaries  }{8}		
	\begin{answer} 
		Given that the propabilites and variances of the two classes are equal, the decision boundary should only be influenced by the two means, sitting centered between them.(x represents the boundary)
		\begin{equation}
		 (x-\mu_1)^2 = (x-\mu_2)^2 \\
		x = \frac{ (x-\mu_1)^2 = (x-\mu_2) }{Q * (\mu_1-\mu_2)}\\
		\end{equation}
		\begin{equation}
		If  \mu_1 = \mu_2 , No decision Boundary 
		\end{equation}

		\begin{equation}
		else: x =\frac{\mu_1 + \mu_2}{2}
		\end{equation}
			
	\end{answer}
		
	\end{question}
	
	%----------------------------------------------
	\begin{question}{Different Misclassification Cost }{8}		
	\begin{answer} 
		Given that wrongly identifying a case of $C_2$ as $C_1$ is more costly than the other way around, the Decision Boundary has to be moved towards $C_1$, causing samples to be more often identified as $C_2$ \\
		$ \mu_1>0; \mu_1=2*\mu_2; \delta_1=\delta_2; p(C_1)=p(C_2) \\\\
		4(2\pi *\delta_1^2)^{ \frac{-1}{2} } * \exp( -\frac{(x-\mu_1)^2}{2\delta_1^2} ) * p(C_1) 
		=(2\pi *\delta_2^2)^{ \frac{-1}{2} } * \exp( -\frac{(x-\mu_2)^2}{2\delta_2^2} ) * p(C_2)
		\\
		\\
		\log(4) + \frac{(x-\mu_2)^2}{2\delta_2^2} = \frac{(x-\mu_2)^2}{2\delta_2^2}
		\\
		\\
		2\mu_2^2 * x -3\mu_2^2 = -\log(4) * \delta_2^2
		\\
		\\
		x = \frac{3\mu_2^2 - \log(4) * 2\delta_2}{2\mu_2^2}
		$
	\end{answer}
		
	\end{question}
	
\end{questions}

